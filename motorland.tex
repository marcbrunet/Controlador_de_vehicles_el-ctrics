\chapter{Motostudent}
\label{chap:motostudent}

El Motostudent és un projecte educatiu que tracta en dissenyar i construir una moto elèctrica. Un cop s'implementi la moto es realitzaran una sèrie de proves en el circuit de Motorland a Aragó. No és el mateix ordre de magnitud que un patinet elèctric pero ha sigut una gran experiència per veure els problemes i les solucions en els vehicles elèctrics.

En el tema de les bateries he pogut viure en primera tot lo fàcil que sembla al principi i lo complica que és a la realitat. El fet de dissenyar una bateria per una aplicació en concret, construir-la i mantenir-la amb sistemes de BMS ha suposat una millora dels meus coneixements en tot aquest camp. Hi ha molts aspectes a tenir en compte; les cel·les per fabricació no són idèntiques i això provoca que s'hagin de tenir en compte a l'hora de connectar-les ja que poden tenir diferències de voltatge que es dissipen en el moment de la connexió. S'han de pensar on i com es posen els termistors per veure les diferències de temperatura en la bateria, i com es refredarà en el cas que sigui necessari. També vaig veure la realització de les connexions per tal de que la intensitat quedés repartida en dos packs de cel·les, una de 24 cel·les i una altra de 4, formant la bateria total de 28. S'havia de tindre en compte que la intensitat havia de quedar dividida en paral·lel entre els dos packs i per tant calia repartir la intensitat de les 24 cel·les a aquestes 4 per tal de balancejar al complert la bateria. 
%en el tema de les baterias he pugut contatar lo facil que semble i lo complicat que es el fet de disnay una baterya per una aplicaico en concaret costryila i mentanirla amb sistemas de BMS, hi han tot de especters a tenir en conte, es a dir les celes per ontrucio no son identecac i axo proboca que agis de nar en conte a lora de connectarles ja que poden tenir petites difarencias de voltage que es disiparen el el moment de la connexio, san de pensar on i com es posen els termitors per veure las difarencias de temperatura en la bateria, i com es refarada en el cas que sigui nesesari, com es realisaren les connexions per tal de que la intenciat en dividexon en el paralel en cuestrio, es d dir la bateria no eran 28 celens en seria sino que tambe en tenin 4 en paralel i axo probocaba que sagen de repartit la intenciat totl en aquets 4.\smallskip

He pogut veure els riscs que comporta treballar amb bateries d'alt voltatge, ja que no disposen de cap sistema de protecció per tal de tallar la potència quan s'estiguin manipulant. En una empresa per exemple, si es produeix un curtcircuit existeixen mesures de protecció com per exemple magnetotèrmics diferencials que evitant que algú es pugui quedar enganxat al corrent. Al treballar amb bateries no es van tenir cap d'aquests sistemes de control per tant es van haver d'aplicar unes grans mesures de protecció i anar amb molt de compte al manipular-les. 

També he pogut veure els sistemes d'activació de la potència. En el cas de la bicicleta no s'han tingut en compte ja que el voltatge i la potència són relativament baixos. En la moto es disposava d'un sistema que determinava si connectava o no la bateria a la resta de la moto o si permetia els pas de corrent del carregador a la bateria. Aquest sistema de control activava el controlador del motor, el qual li subministrava la potència a ell i el BMS controlava el carregador, de forma que abans de permetre el pas de potència comprovaven l'estat de la bateria.
%sitemas de activacio de la potencia, en el cas del patinet no san tingut en conte ja que el voltage i la potencia son reletivament baixes pero en la moto disposava de un sistema que daterminava si connectava o no la baterya al resto de la moto o si permetia el pas de corrent del carregador a ala baterya, aquets sistemas de contorl anava activar el el controlador de motor el que li sumnistrava la potencia a ell i el bms controlaba el caregador, de forma que avans de permentre el pas de potencia comprovaven el estat de la baterya.\smallskip 

Per controlar la moto es va anar per un model de convertidor comercial, un controlador per a motors brushless ja que les potències podien arribar als 48KW. Això requeria d'una electrònica i control molt estable. El control havia d'estar alimentat per dues fonts separades; una que era la bateria principal i l'altre la bateria de control, la qual només activava els sistemes de càlcul del BMS i el Driver per tal de determinar si estava o no estava a punt per controlar el motor.

En el control de la bateria vam optar per un orionBMS que és un model comercial de BMS el qual ens permet controlar l'anivellació de les cel·les i el seu voltatge i intensitat que es drena de la bateria. A part de disposar de dues interfícies de communicació canbus per tal de controlar el carregador de bateries i treure informació.

%ziban??????
El carregador era en ziban el qual podia injectar fins a 50A a la bateria, els seus paràmetres d'intensitat, voltatge i ones de càrrega anaven controlats per una interfície canbus, en el qual el BMS i el carregador s'havien d'entendre i demanar els paràmetres òptims per a la càrrega.
%el carragador era en ziban el cual podei ingectar fins a 50A a la beterya, els seus paremntres de intenciat voltage i odes de carrega anavan controlats per un inteface canbus, en el cual el bms i el caragador savien de entendre i demanr els paramentes optims per a la carrega\smallskip

El blender era un sistema que es dedicava a la vigilància de que les fonts de voltatge LOW i HIGH no estiguessin connectades per tallar la bateria en cas que passés. És un sistema de seguretat que es basa en que el sistema de potència sigui un sistema totalment aïllat tant en el positiu com en el negatiu, i que no tingui sortida en cap massa.

Un cop obtinguts tots els resultats ens vam reunir tot l'equip i vam fer una pluja dels problemes i aspectes a corregir sobre la moto. És allà on hem vaig adonar de la dificultat del projecte que estàvem portant entre mans. Durant el període del disseny vam estar cuidant tots els elements per a garantir una moto de qualitat. En aquell moment semblava que tot anava sobre rodes, però després al dia d'avui m'adono que vam passar per alts petits detalls que al final van portar molts mal de caps. Aspectes que no vam tenim en compte com el soroll emès per un corrent de 500A a 110V passat per cables no apantallats, fent inservible l'electrònica. El BMS no detectava els voltatges de les cel·les i tallava la bateria, ja que saltaven els sistemes de seguretat només pel soroll emès. En futures millores realitzarem tot el cablejat apantallat i controlarem la direcció i els retorns dels camps per tal d'anular al màxim aquests efectes. El sistema d'aïllament ens tallava la bateria en el procés de càrrega ja que, el canbus del carregador es connectava amb el canbus de la moto, on l'aïllament no era suficientment alt. Per això es va haver de puntejar el sistema d'aïllament per poder carregar la moto.

Les comunicacions eren inútils en el moment de càrrega o descàrrega de la bateria ja que el soroll provocat i el no anar amb cables apantallats ni trenats provocava errors a les comunicacions.

Trobo que ha estat una experiència molt educativa, molt bona en tots els sentits, pero trobo que la nostra universitat no ho té en compte, i es fan petites activitats com el futbol. És un projecte on s'aprén com funcionen les coses i com la teoria no quadra per als petits detalls.

