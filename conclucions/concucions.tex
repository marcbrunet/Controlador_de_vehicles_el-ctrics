\chapter{Conclusions}
\label{chap:conclusions}

Després de la realització del projecte i amb tot l'estudi previ que hem realitzat em trobo en posició de determinar que no era conscient de la complexitat que suposa controlar el voltatge i la quantitat de paràmetres que s'han de tenir en compte per tal de controlar aquestes magnituds. Per tant l'abast del projecte inicial ha estat reduït ja que tenia al cap el disseny com la construcció d'un controlador de vehicles elèctrics. Encara hi han aspectes de l'electrònica de potència que no he assimilat completament i no podria dissenyar un controlador estable sense ajuda. He pogut aprofundir en el control de motors i els tipus de controls que existeixen, en el món dels vehicles elèctrics i cap on va el món del petit VE.

A l'acabar el projecte col·laborant amb MotoStudent veig que la complicació també la tindrem de punts que no he tingut en compte com les emissions dels cables de potència i la seguretat en sistemes de protecció de les bateries, és a dir, mantenir els circuits aïllats i amb el control de les emissions per tal de no espatllar les mesures. Així que em queda seguir treballant per entendre els motors elèctrics per poder-me dissenyar un petit vehicle elèctric.
