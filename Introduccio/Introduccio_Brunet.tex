\chapter{Introducció}
\label{chap:intro}

%Petita introducció

\section{Elecció del TFG}
He escollit aquest projecte després de realitzar l'assignatura optativa \newline d'energies renovables, eòlica i fotovoltaica. El fet de fer aquesta assignatura em va fer veure que tenia molt poc coneixement en el món de l'electrònica de potència. Aquest va ser el factor determinant per buscar un projecte relacionat amb aquesta l'electrònica, tot buscant un tema interessant que estengui els meus coneixements per al món laboral. A més a més ja que sóc un aficionat als esports, m'ha semblat un tema molt interessant intentar aplicar aquesta electrònica de potència en els vehicles elèctrics que jo he fet servir, com podria ser una bicicleta o el monopatí. D'aquesta manera vaig decidir que el meu projecte aniria relacionat amb la transmissió de l'energia elèctrica d'unes bateries i el funcionament d'un motor elèctric.

L'altre punt que m'ha cridat molt l'atenció és que ara a l'actualitat tenim tecnologia per tot arreu, miris on miris pots veure un munt d'aparells elèctrics que ens rodegen constantment. Molts d'aquests aparells funcionen mitjançant bateries i cada cop més, les bateries són millors i millors. Els propis fabricants es preocupen molt de l'evolució de les bateries degut a que en el futur molts més aparells funcionaran amb bateries. És per això que cal tenir en compte que el món de les bateries és un món que en uns anys serà una eina imprescindible per a la tecnologia. És molt interessant conèixer els tipus de bateries, els seus problemes i avantatges i com s'han de gestionar. Com evolucionaran al futur i quines hi havia al passat. Amb l'elecció d'aquest treball de recerca també puc ampliar els coneixements en aquest món.

Una vegada tenim el coneixement de la potència i de l'electroquímica de les bateries, i com a punt principal d'aquest treball de recerca es veurà com transferir aquesta potència en els motors elèctrics, per tal de poder aprofitar aquesta energia que hi ha acumulada a la bateria tot tenint en compte el sentit contrari, on ja s'estaria parlant de conceptes que apareixen en els motors elèctrics com és la frenada regenerativa, que consisteix en l'energia que genera el motor quan es frena. Crec que el concepte de poder transformar energia a moviment i a la inversa em pot donar coneixements molt més amplis del funcionament de l'electrònica de potència. 

\section{Motivacions a l'hora de fer un controlador de vehicles elèctrics}
Per fer aquest projecte ens va motivar l'increment de vehicles elèctrics que hi ha actualment al mercat, ja poden ser patinets, monopatins o bicicletes que poc a poc van apareixent més i més en la societat. Aquests permeten fer viatges de petites distàncies dintre d'un nucli urbà, o augmentar la facilitat en que algunes persones realitzen activitats esportives com l'increment que s'està veient en les EBIKES. El fet d'haver emprat també aquests tipus de vehicles li dóna més punts encara al fet de voler fer aquest projecte.
    
Encara que el meu interès es trobi en l'electrònica de potència hi ha una part que tracta l'electrònica donada al nostre grau, molt més centrada en el control de la bateria. Aquesta part tracta a uns nivells elèctrics molt inferiors als que es tracten al motor, però al cap i a la fi, és d'on surt tota la potència per a què pugui ser transformada en la força que fa que el motor giri. Per tant penso que tota l'electrònica que hi ha darrere, la gestió que ha de realitzar un microcontrolador, com es tracta la dissipació, són temes també molt interessants i que em desperten una gran curiositat.

Per últim, el repte que suposa dissenyar un controlador per a vehicles elèctrics, el qual pugui arribar a ser escalable per a poder afegir diferents mòduls per a gestionar un major nombre de bateries o per gestionar un motor d'una major potència, fa que no hi hagi cap mena de dubte en el treball de recerca el qual vull realitzar. Donat que és un camp que s'ha tractat molt poc i el poc coneixement d'aquest, caldrà veure si a la realitat serà possible aconseguir aquest repte. 

\section{Objectius a assolir}

La nostra idea principal és un sistema modular que es basa en dos grans blocs; el primer consisteix en un controlador capaç de gestionar un cert nombre limitat de bateries. Aquest mòdul ha de poder controlar i balancejar les cel·les de les bateries per tal de prolongar la vida útil d'aquestes, millorar el seu rendiment i controlar que el sistema funcioni de forma segura. De forma resumida vindria ser la mare que cuida de les cel·les per a que no els hi falti ni els hi sobri res. Aquest mòdul seria la nostra la font d'energia en forma de corrent.  Aquesta energia seria emprada per alimentar el segon gran bloc, que consisteix en el tractament d'aquesta energia per tal de controlar un motor, i per tant, convertir l'energia elèctrica en energia cinètica en forma de moviment. Aquest mòdul també ha de tindre en compte el concepte de la frenada regenerativa que succeeix quan el vehicle frena. En aquest moment el motor no està consumint energia, sinó que n'està generant. Aquesta energia retorna cap a tot aquest controlador que si no es té en compte molt possiblement cremi el circuit elèctric. És un projecte molt ambiciós ja que no estem especialitzats en la matèria de l'electricitat de potència, caldrà veure fins on serem capaços d'arribar i els coneixements que aprendrem durant aquest procés.

Ens plantegem el projecte com una forma d'aprendre uns \newline coneixements de bateries, motors i sistemes que s'encarreguen de controlar aquests, per d'alguna forma ser molt més conscients del mercat disponible a l'actualitat, del qual ja es poden trobar treballs per l'estil els quals ens ajudaran i facilitaran tota la recerca prèvia del projecte. El fet de poder trobar treballs pujats per la comunitat\footnote{Comunitat de desenvolupadors de software i hardware obert que comparteixen els seus projectes per a fomentar la idea de créixer amb els coneixements dels demés} per tal de fent-se una idea de que tenim disponible i que es pot aplicar a aquest camp, és a dir, valorar d'una forma crítica les solucions proposades per tal de veure si tenim una bateria perfecta o quin és el millor sistema de control disponible per a la bateria, quins tipus de motors són els més adequats i com es controlen. Comentarem algunes de les solucions comercials però si disposem de solucions de la comunitat seran les que ens expliquin d'una forma més detallada, ja que al ser lliures, pots veure a fons cadascuna de les parts.

\section{Aplicacions que volem que tingui}
Ens plantegem un projecte en el que partim de molts camps \newline desconeguts, per tant l'aplicació principal que volem que tingui el nostre projecte és aconseguir assimilar la major quantitat de concepte del funcionament dels petits vehicles elèctrics. Partint d'aquesta base i sabent el repte que conforma, es plantejarà el prototipatge d'un controlador de vehicles elèctrics ja sigui fet des de 0 o partint d'un projecte similar trobat a Internet. Tenint clars els conceptes es pot arribar a crear un controlador de vehicles elèctrics modificat al gust d'un projecte.

També cal tenir en compte que la majoria de conceptes d'electricitat o electrònica de potència s'hauran d'aprendre en aquest treball, per tant és molt probable que amb l'abast d'aquest projecte el prototip que \newline s'explicaria seria el d'un ja realitzat.

Una vegada tinguem els coneixements de com es controla i quin tipus de control tenim, valorarem la complexitat del control i com s'aplica també aquesta potència en tot moment.

Una de les aplicacions principals que voldríem que tingués seria per a un patinet o una bicicleta elèctrica, per tal de poder donar-li un ús urbà, ja que són els dos vehicles més còmodes per a entorns urbans. Seria molt interessant que en cas de ser necessari, tenir l'habilitat de canviar els paràmetres del control, per tal d'adaptar el funcionament a l'aplicació de forma que es pugui adaptar a les necessitats completes de l'aplicació, sempre i quant aquests canvis siguin únicament de software, ja que un canvi de hardware suposaria un canvi de disseny.
